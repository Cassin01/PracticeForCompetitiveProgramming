\documentclass[12pt]{jarticle}
\setlength{\textwidth}{170mm}
\setlength{\textheight}{260mm}
\setlength{\oddsidemargin}{-5mm}
\setlength{\topmargin}{-25mm}
\usepackage[dvipdfmx]{graphicx}
\usepackage{ascmac}

\begin{document}
\begin{center}
  学籍番号: 4617041 氏名: 白江涼雅
\end{center}

{\LARGE 課題2 ハミング符号を用いた誤り訂正}

\section{目的}
実際にハミング符号のスクリプトをC言語で生成し,ハミング符号についての理解を深める.

\section{実験装置}
\begin{itemize}
 \item windows Xシリーズ
 \item Visualstdio2013
\end{itemize}

\section{結果}
最後に示すソースコードを動かすと,以下のような結果になった.
\begin{center}  
\begin{screen}
\begin{verbatim}
情報語:  1  1  0  1
符号語:  1  1  0  1  0  1  0
誤らせるのは何回目(1~7)?:4
受信語  1  1  0  0  0  1  0
シンドローム  0  1  1
4ビット目を反転させます
推定後符号語  1  1  0  1  0  1  0
符号語のビット誤り数は0です
\end{verbatim}
\end{screen}
\textbf{図1:ハミング符号のプログラム実行結果}
\end{center}

\section{検討事項}
\begin{enumerate}
\item なぜ(7,4)ハミング符号は1個誤りを訂正できるか.\\
  最後にのせているソースコードにあるように生成行列G=[$I_k A$]を作った.Gに合わせてH=[$A^T I$]もつくった.GとHの関係は$HG^T = HG^T = 0$ となる値をとる.そうすることで、Gに情報語$i$をかけてたとしても$iG^TH=0$とすることができるためである.\\
  iに右からGをかけて送信符号xを作り,誤り符号eを付け加えた値を受信信号yとする.つまりyは$y=iG + e$となっている.\\
  受信信号を受け取った側は$y$に対し$H^T$をかけ、シンドローム$s$を得られる.つまりsは$s=(iG + e)H^T$である.$iGH^T=0$であるため、残るのは$eH^T$である.もしsが[0 0 0]ならば誤りはなく,誤りがあるのなら誤りに対応するシンドロームがでる.\\
  といった理由で1個誤りは訂正できる.ここではシンドロームは3行1列であるため,1元で8通りのシンドロームが表わせ、0もしくは1個誤りは1元的で合計8通りであるため、eは一意に定まる.\\
 すなわち、誤りの数がシンドロームのパターンとすべて異なるように表現できるので、誤りを特定できる.

\item 2個以上の誤りが発生するとどうなるか.2個,3個, ...とするとどうなるか.\\
  (7,4)ハミング符号で2個以上の誤りが発生してしまうと,sの3行1列では表しきれないため,想定される何通りかまでは絞れるが,一意には定まらなくなってしまう.

\end{enumerate}

\section{ソースコード}
\begin{center}  
%\begin{screen}
\begin{verbatim}

#include <stdio.h>
#include <random>
#define gyo 7 //ハミング符号
#define retu 4 //ハミング符号
#define k 3 //シンドロームの長さ

int main(){
	int G[gyo][retu];//生成行列
	int H[gyo][k]; //検査行列
	int w[retu];  //情報系列
	int x[gyo];  //送信系列
	int y[gyo];  //受信系列
	int e[gyo];  //誤り符号
	int s[gyo]; //シンドローム生成
	double ran;
	int i,j;
	int tmp;
	int miss;

	//乱数発生準備
	std::mt19937 mt(41);
	std::uniform_real_distribution<double> r_rand(0.0, 1.0);

	
	//生成行列で必要な任意にきめるところの生成
	G[0][0] = 1; G[0][1] = 0; G[0][2] = 0; G[0][3] = 0;
	G[1][0] = 0; G[1][1] = 1; G[1][2] = 0; G[1][3] = 0;
	G[2][0] = 0; G[2][1] = 0; G[2][2] = 1; G[2][3] = 0;
	G[3][0] = 0; G[3][1] = 0; G[3][2] = 0; G[3][3] = 1;
	G[4][0] = 1; G[4][1] = 1; G[4][2] = 1; G[4][3] = 0;
	G[5][0] = 1; G[5][1] = 1; G[5][2] = 0; G[5][3] = 1;
	G[6][0] = 1; G[6][1] = 0; G[6][2] = 1; G[6][3] = 1;

	H[0][0] = 1; H[0][1] = 1; H[0][2] = 1;
	H[1][0] = 1; H[1][1] = 1; H[1][2] = 0;
	H[2][0] = 1; H[2][1] = 0; H[2][2] = 1;
	H[3][0] = 0; H[3][1] = 1; H[3][2] = 1;
	H[4][0] = 1; H[4][1] = 0; H[4][2] = 0;
	H[5][0] = 0; H[5][1] = 1; H[5][2] = 0;
	H[6][0] = 0; H[6][1] = 0; H[6][2] = 1;

	/*
	//生成確認
	for (i = 0; i < retu; i++){
		for (j = 0; j < gyo; j++){
			printf("%3d", G[j][i]);
		}
		printf("\n");
	}
	printf("\n\n");

	for (i = 0; i < k; i++){
		for (j = 0; j < gyo; j++){
			printf("%3d", H[j][i]);
		}
		printf("\n");
	}
	printf("\n\n");
	*/
	//wの生成
	for (i = 0; i < retu; i++){
		ran = r_rand(mt);
		if (ran < 0.5){
			w[i] = 0;
		}
		else{
			w[i] = 1;
		}
	}
	//wの出力
	printf("情報語:");
	for (i = 0; i < retu; i++){
		printf("%3d", w[i]);
	}
	printf("\n");

	//xの生成
	for (i = 0; i < gyo; i++){
		tmp = 0;
		for (j = 0; j < retu; j++){
			tmp += w[j] * G[i][j];
		}
		if (tmp % 2 == 0){
			x[i] = 0;
		}
		else{
			x[i] = 1;
		}
	}
	//xの確認
	printf("符号語:");
	for (i = 0; i < gyo; i++){
		printf("%3d", x[i]);
	}
	printf("\n");

	//誤りeの生成
	for (i = 0; i < gyo; i++){
		e[i] = 0;
	}
	printf("誤らせるのは何回目(1~7)?:"); scanf("%d", &miss);
	e[miss-1] = 1;
	
	//送信行列に誤りeを干渉させ,送信行列をつくる
	for (i = 0; i < gyo; i++){
		y[i] = (x[i] + e[i]) % 2;
	}
	//yの確認
	printf("受信語");
	for (i = 0; i < gyo; i++){
		printf("%3d", y[i]);
	}
	printf("\n");


	//sの生成
	for (i = 0; i < k; i++){
		s[i] = 0;
		for (j = 0; j < gyo; j++){
			s[i] += y[j] * H[j][i];
		}
		if (s[i] % 2 == 1){
			s[i] = 1;
		}else{
			s[i] = 0;
		}
	}
	printf("シンドローム");
	for (i = 0; i < k; i++){
		printf("%3d", s[i]);
	}
	printf("\n");

	//どこ反転させるかの判定
	if (s[0] == 1 && s[1] ==1 && s[2] ==1 ){
		miss = 1;
	}else if (s[0] == 1 && s[1] == 1 && s[2] == 0){
		miss = 2;
	}
	else if (s[0] == 1 && s[1] == 0 && s[2] == 1){
		miss = 3;
	}
	else if (s[0] == 0&& s[1] == 1&& s[2] == 1){
		miss = 4;
	}
	else if (s[0] == 1&& s[1] == 0&& s[2] == 0){
		miss = 5;
	}
	else if (s[0] == 0&& s[1] == 1&& s[2] == 0){
		miss = 6;
	}
	else if (s[0] == 0&& s[1] == 0&& s[2] == 1){
		miss = 7;
	}
	else{
		miss = 0;
	}

	printf("%dビット目を反転させます\n",miss);

	if (y[miss - 1] == 0){
		y[miss - 1] = 1;
	}
	else{
		y[miss - 1] = 0;
	}
	printf("推定後符号語");
	for (i = 0; i < gyo; i++){
		printf("%3d", y[i]);
	}
	printf("\n");
	printf("符号語のビット誤り数は0です\n");

}

\end{verbatim}
%\end{screen}
\textbf{図2:ハミング符号プログラムソースコード}
\end{center}

\section{参考文献}
\begin{thebibliography}{2}
\bibitem{}  実験の際もらったプリント
  \bibitem{} 情報工学実験2 (2018)
  \end{thebibliography}

\end{document}
