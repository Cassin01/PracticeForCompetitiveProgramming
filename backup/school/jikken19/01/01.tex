\documentclass[a4paper, 12pt]{jarticle}
\usepackage{array}
\usepackage[deluxe]{otf}
\usepackage{multirow}
\usepackage{multicol}
\newcommand{\uu}[1]{\underline{ \textbf{#1}}}

\newcommand{\xx}[1]{\textbf{#1}}
\begin{document}
\begin{center}
学籍番号: \underline{4617043},
名前: \underline{神保光洋}
\end{center}

以下の表は, 東京理科大学工学部情報工学科の2年生向けに開講されている情報工学実験Iのスケジュールを、\LaTeX により作成したものである. 作成のためには, 以下のスタイルファイルをusepackageするとよい.

\begin{itemize}
    \item array.sty
    \item otf.sty (ただし, deluxeオプションをつける)
    \item multirow.sty
    \item multicolumn.sty
\end{itemize}

\[
\begin{array}{|c||c|c|c|c|c|c|c|c|c|c|c|c|}
    \hline
    実験日 & \xx{G1} & \xx{G2} & \xx{G3} & \xx{G4} & \xx{G5} & \xx{G6} & \xx{G7} & \xx{G8} & \xx{G9} & \xx{G10} & \xx{G11} & \xx{G12} \\

    \hline
    \hline

    4 月 15 日 & \multicolumn{12}{c|}{ガイダンス,  $\LaTeX$ 演習(\underline{0})} \\
    \hline
    \hline
    実験日 & \xx{G1} & \xx{G2} & \xx{G3} & \xx{G4} & \xx{G5} & \xx{G6} & \xx{G7} & \xx{G8} & \xx{G9} & \xx{G10} & \xx{G11} & \xx{G12} \\ \hline
    4 月 22 日 & \multirow{3}{*}{\uu{2}} & \multirow{3}{*}{\uu{2}} & \multirow{3}{*}{\uu{2}} & \multirow{3}{*}{\uu{3}} & \multirow{3}{*}{\uu{3}} & \multirow{3}{*}{\uu{3}}
                & \multirow{2}{*}{\uu{1}} & \multirow{2}{*}{\uu{1}} & \multirow{2}{*}{\uu{5}} & \multirow{2}{*}{\uu{5}} & \multirow{2}{*}{\uu{4}} & \multirow{2}{*}{\uu{4}} \\
    5 月 06 日 & & & & & & & & & & & & \\ \cline{8-13}
    5 月 13 日 & & & & & & & \multirow{2}{*}{\uu{4}} & \multirow{2}{*}{\uu{4}} & \multirow{2}{*}{\uu{1}} & \multirow{2}{*}{\uu{1}} & \multirow{2}{*}{\uu{5}} & \multirow{2}{*}{\uu{5}} \\ \cline{1-7}

    5 月 20 日 & \multirow{3}{*}{\uu{3}} & \multirow{3}{*}{\uu{3}} & \multirow{3}{*}{\uu{3}} & \multirow{3}{*}{\uu{2}} & \multirow{3}{*}{\uu{2}} & \multirow{3}{*}{\uu{2}}
                & & & & & & \\ \cline{8-13}
    5 月 27 日 & & & & & & & \multirow{2}{*}{\uu{5}} & \multirow{2}{*}{\uu{5}} & \multirow{2}{*}{\uu{4}} & \multirow{2}{*}{\uu{4}} & \multirow{2}{*}{\uu{1}} & \multirow{2}{*}{\uu{1}} \\
    5 月 03 日 & & & & & & & & & & & & \\ \hline \hline

    6 月 10 日 & \multirow{2}{*}{\uu{1}} & \multirow{2}{*}{\uu{1}} & \multirow{2}{*}{\uu{5}} & \multirow{2}{*}{\uu{5}} & \multirow{2}{*}{\uu{4}} & \multirow{2}{*}{\uu{4}}
                & \multirow{3}{*}{\uu{1}} & \multirow{3}{*}{\uu{1}} & \multirow{3}{*}{\uu{2}} & \multirow{3}{*}{\uu{2}} & \multirow{3}{*}{\uu{3}} & \multirow{3}{*}{\uu{3}} \\
    6 月 17 日 & & & & & & & & & & & & \\ \cline{1-7}

    6 月 24 日 & \multirow{2}{*}{\uu{4}} & \multirow{2}{*}{\uu{4}} & \multirow{2}{*}{\uu{1}} & \multirow{2}{*}{\uu{1}} & \multirow{2}{*}{\uu{5}} & \multirow{2}{*}{\uu{5}} & & & & & & \\ \cline{8-13}
    6 月 01 日 & & & & & &
            & \multirow{3}{*}{\uu{3}} & \multirow{3}{*}{\uu{3}} & \multirow{3}{*}{\uu{3}} & \multirow{3}{*}{\uu{2}} & \multirow{3}{*}{\uu{2}} & \multirow{3}{*}{\uu{2}} \\ \cline{1-7}

    6 月 08 日 & \multirow{2}{*}{\uu{5}} & \multirow{2}{*}{\uu{5}} & \multirow{2}{*}{\uu{4}} & \multirow{2}{*}{\uu{4}} & \multirow{2}{*}{\uu{1}} & \multirow{2}{*}{\uu{1}} & & & & & & \\
    6 月 15 日 & & & & & & & & & & & & \\ \hline \hline
    7 月 22 日 & \multicolumn{12}{c|}{予備日} \\ \hline

\end{array}
\]

\begin{flushright}
    (注) 表中の下線付数字は課題番号である.
\end{flushright}

\end{document}
